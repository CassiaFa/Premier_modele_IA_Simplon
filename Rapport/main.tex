\documentclass[french]{article}
\usepackage[utf8]{inputenc}
\usepackage[T1]{fontenc}

\usepackage{natbib}
\usepackage[vmargin=3cm,left=4cm,right=4cm]{geometry}

\usepackage{amsmath}

\usepackage{babel}

\usepackage{graphicx}
\usepackage{caption} 
\captionsetup{justification=centering}
\usepackage{subcaption}
% \usepackage{hyperref}
\usepackage[hidelinks]{hyperref}

\begin{document}

%###############################################
\begin{titlepage}

\newcommand{\HRule}{\rule{\linewidth}{0.5mm}} % Defines a new command for the horizontal lines, change thickness here

\center % Center everything on the page
 
%----------------------------------------------------------------------------------------
%	Section Titre
%----------------------------------------------------------------------------------------
\HRule \\[0.4cm]
\vspace{1cm}
{ \huge \bfseries Premier modèle IA}\\ % Title of your document
\vspace{1cm}
\HRule \\[1cm]
 
%----------------------------------------------------------------------------------------
%	Section auteur
%----------------------------------------------------------------------------------------
\vspace{1cm}

\Large \today

\vspace{3cm}

\begin{minipage}{0.4\textwidth}
\begin{center}
\Large \textbf{Auteurs :}\\
\vspace{0.5cm}
Fabio \textsc{Cassiano}
\end{center}
\end{minipage}

\vspace{5cm}

\begin{figure}[!ht]
    %\hspace*{-0.5cm}
	\includegraphics[height=0.1\columnwidth]{images/logo/logo_simplon.png}
	\hspace*{0.5cm}
	\includegraphics[height=0.12\columnwidth]{images/logo/logo_Isen.png}
	\hspace*{0.5cm}
	\includegraphics[height=0.1\columnwidth]{images/logo/logo_microsoft.jpg}
\end{figure}

\vfill

\end{titlepage}

\newpage

\tableofcontents

\newpage

\section{Rappel sur la régression linéaire}

La régression linéaire est une méthode statistique qui a pour objectif de trouver une relation entre une variable cible (\textit{target}) à partir d'une variable dite descriptive. La régression linéaire peut-être appliqué sur différents types de modèles qui vont être abordés dans les sous-section ci-dessous. 

\subsection{Modèle simple}

Le modèle linéaire simple, est un modèle basé sur une fonction affine. Cette fonction a pour équation mathématique :

\begin{center}
    $f(x) = ax + b$
\end{center}

\noindent Dans cette équation $f(x)$ représente la valeur cible, soit celle qu'il faut prédire. La valeur $x$ correspond à la variable descriptive à partir de laquelle on pourra obtenir $f(x)$. Les valeurs a et b correspondent aux coefficients, qui représentent respectivement la pente et l'ordonnée à l'origine. 

\subsection{Modèle multiple}

Le modèle linéaire multiple est une extension du modèle linéaire simple, qui permet la prédiction du variable cible à partir de plusieurs variables descriptives. Ce modèle a pour équation :

\begin{center}
    $f(x) = ax_{1} + bx_{2} + c $
\end{center}

\noindent Tout comme pour l'équation précédente $f(x)$ représente la valeur cible, soit celle qu'il faut prédire. Les valeur $x_{1}$ jusqu'à $x_{n}$ représentent aux différentes variables descriptives de notre jeu de données.

\subsection{Modèle polynomiale}

Concernant le modèle polynomiale, il correspond à une équation à une inconnue, avec un polynôme de degré $n$. Mathématiquement cela se traduit de la manière suivante :

\begin{center}
    $f(x) = ax^{n} + bx^{n-1} + c $
\end{center}

\noindent Par exemple pour un polynomial de degré 2 l'équation est :

\begin{center}
    $f(x) = ax^{2} + bx + c $
\end{center}

\newpage

\section{Fonction sous python}

Ces différent modèle peuvent être retranscrit sous python. Pour facilité leur modélisation on peut les écrire modèles sous forme matricielle.

\subsection{Modèle simple}

\noindent L'écriture matricielle du modèle linéaire simple est donc la suivante :
\begin{align*}
    F = X \cdot \theta
\end{align*}

\noindent avec, 
$$X = \begin{bmatrix}
    x^{(1)} & 1 \\
    \vdots & \vdots \\
    x^{n} & 1
\end{bmatrix}$$

\noindent et, 
$$\theta = \begin{bmatrix}
    a \\
    b
\end{bmatrix}$$

\noindent où $X$ est la matrice qui contient la variable descriptive $x$, sous forme de colonne. Une colonne composé uniquement de $1$ est également inclut dans $X$, ce qui correspond au multiplicateur du coefficient \textit{b}. Enfin la matrice $\theta$ regroupe les coefficients \textit{a} et \textit{b}.
\subsection{Modèle multiple}

\subsection{Modèle polynomiale}

\section{Résultats}
\subsection{Modèle simple}

\subsection{Modèle multiple}

\subsection{Modèle polynomiale}

\newpage


\end{document}
